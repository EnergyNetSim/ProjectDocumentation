
\section{Präsentation und Auswertung der Ergebnisse} \label{sec:Erg}
Für die Darstellung der Ergebnisse des Projekts soll zunächst der entwickelte Algorithmus zur Berechnung des Stromverbrauchs im optimierten Netz beschrieben werden. Anschließend wird in \ref{subsec:ErgSoftware} ausführlich auf die erstellte Software „EnergyNetSim“ und deren Benutzung eingegangen, bevor in \ref{subsec:ErgDiskussion} eine Diskussion der Ergebnisse und Erkenntnisse folgt.

\subsection{Algorithmus} \label{subsec:ErgAlg}
Wie in Kapitel \ref{subsec:MethAlg} dargestellt, wurde ein Algorithmus entwickelt, der es erlaubt, Netzlast (Traffic) dynamisch durch ein definiertes Netz zu routen und anschließend nicht benötigte Geräte und Verbindungen abzuschalten.
Grundlegend ist das Ziel, für jeden Zeitabschnitt die beste erzielbare Lösung zur stromsparenden Bewältigung (Durchleitung) des Traffic von dem Eingangspunkt in das Netz (Quelle) bis zum Austrittspunkt (Senke) zu berechnen. Zur Vereinfachung wurde die Annahme getroffen, dass über die Dauer eines Zeitabschnittes (einer Iteration) die durch den Algorithmus getroffenen Routing-Entscheidungen und die Wahl der abzuschaltenden Hardware gleich bleiben. Diese werden je Iteration initial einmal neu berechnet. Durch die Verkürzung der Iterationsdauer und der dazu passenden Netzlast-Daten ließe sich bei Bedarf mit geringem Aufwand ein genaueres, verfeinertes Modell durchrechnen, so dass ein realistischeres Resultat zu erwarten ist.


\begin{figure}[!ht]
	\centering
	\includegraphics[width=0.6\textwidth]{1Berechnung_elektr_Stromverbrauch}
	\caption{Programmablaufplan zur Anwendungslogik}
	\label{fig:1Berechnung_elektr_Stromverbrauch}
\end{figure}


Je durch den Anwender in der Datenbank hinterlegtem statischen Netz wird eine Datenreihe bestehend aus einem Stromverbrauchswert je Iterationszeitraum berechnet.
Um diese Einzelwerte zu berechnen, werden zuerst die Lasten aus jedem Netzlastprofil-Eintrag auf möglichst effizient auf die Netzwerkhardware und Verbindungen projiziert, und danach abschließend abhängig von den berechneten Geräteauslastungen die Teil-Stromverbrauchswerte addiert.


Mit dieser groben Vorgehensweise lässt sich das gesuchte Ergebnis je Modellnetzwerk inklusive des zeitlichen Verlaufs errechnen. Allerdings fehlt dazu noch die Antwort auf die Frage, wie die effizienten Routing-Entscheidungen getroffen werden können. Der folgende Teil des Algorithmus beschreibt eine möglichen Lösung:


Um zu entscheiden, welches Routing für die einzelnen Netzlastprofil-Einträge je Iteration das beste Resultat liefert, wird je möglicher Route eine dynamische Penalty (Kosten-Faktor) berechnet. Anschließend wird für das Netzlastprofil die Route mit der niedrigsten errechneten Penalty, welche die Datendurchsatzgrenzen der Hardware und somit der einzelnen Verbindungen nicht überschreiten, als beste Route angenommen. Diese Kategorie von lokalen Lösungsverfahren nennt man Greedy Algorithmus. Dieser liefert nicht unbedingt das insgesamt beste Ergebnis, sondern berechnet nur das lokale Optimum für das betrachtete Teilproblem. Um eine schnelle praktische Ausführungszeit auch bei Berechnungen mehrerer Vergleichsnetze mit kurzen Iterationsintervallen zu erreichen beinhaltet dieser Algorithmus intelligente Entscheidungsfunktionen um schlechte Routen frühzeitig zu ignorieren und nicht zielführende Berechnungen, wie sie beispielsweise bei Schleifenbildung auftreten, zu verhindern.
\begin{figure}[ht]
	\centering
	\includegraphics[width=0.6\textwidth]{2Routingentscheidung}
	\caption{Programmablaufplan zum übergreifenden Anteil des Routing-Algorithmus}
	\label{fig:2Routingentscheidung}
\end{figure}

Da die zur Verbindungsbewertung verwendete Penalty mehreren Faktoren wie Latenz, elektr. Stromverbrauch, Kapazität der Verbindung und auch An-/Aus-Status der Hardware berücksichtigen muss, und diese einzelnen Faktoren je nach Verwendungszweck des Netzes unterschiedliche Gewichtung haben, müssen die einzelnen Anteile mit vom Anwender der Simulationssoftware festgelegten Gewichtungsfaktoren multipliziert werden. Damit kann der Nutzer die Netzsimulation auf seine Anforderungen ansatzweise anpassen.
\begin{figure}[ht]
	\centering
	\includegraphics[width=0.6\textwidth]{3Rekursives_Routing}
	\caption{Rekursiver Anteil des Routing-Algorithmus}
	\label{fig:3Rekursives_Routing}
\end{figure}


Alles in allem ist ein Regelsatz zur Penaltyberechnung je Iteration notwendig, an Hand welchem die Algorithmus-Implementierung jeweils versucht das lokale Optimum zum Routingproblem als Lösung zu finden.
Der folgende Regelsatz stellte sich beim logischen durchspielen einer solchen Simulation als Grundlage heraus:
\begin{itemize}
	\item Zum Anfang jeder neuen Iterations-Zeitphase sind alle Geräte und alle Ports soweit diese dies unterstützten ausgeschaltet
	\item Das Aktivieren eines Devices kostet Penalty (hoch)
	\item Das Aktivieren eines Ports kostet Penalty (niedrig)
	\item Das Ausschalten des Port-Energiesparmodus kostet keine Penalty, um vermeidbares aktivieren anderer Hardware / Ports zu vermeiden. Energiesparmodus-Capability zählt lediglich zur Berechnung des Stromverbrauchs.
	\item Um niedrige Paketlaufzeiten durch das Netz sicherzustellen, kostet jeder Hop / genutzte Verbindung eine weitere Penalty (hoch)
\end{itemize}
Einzelne Verbindungen haben zusätzlich zu dem dynamischen Anteil außerdem jeweils eine feste Penalty basierend auf Verstärkeranzahl + Länge -> statischer Wert in Datenbank.

% Diskussion des Ergebnisses und der Schwierigkeiten bei der Implementierung werden im nächsten Kapitel berücksichtigt

%TODO: getroffene Annahmen und Verallgemeinerungen erläutern

\subsection{Die erstellte Software aus Ergebnissicht}\label{subsec:ErgSoftware}
Im Rahmen der Software-Entwicklung wurde das Java-Programm „EnergyNetSim“ mitsamt Anbindung an eine MySQL-Datenbank realisiert. Die folgenden Unterkapitel beschreiben die erzielten Resultate und geben Hinweise auf Installation und Benutzung der Software für Endanwender.

\subsubsection{Ergebnisse der Software-Entwicklung}
Die erstellte Software „EnergyNetSim“ stellt ein Rahmenwerk zur Verfügung, in das die in den Kapiteln
\ref{subsec:MethAlg} und \ref{subsec:ErgAlg} beschriebenen Algorithmen zur Simulation von Netzauslastung, Datenverkehr zwischen zwei Netzknoten, dynamischem Routing und effizienzsteigernden Energiesparmechanismen eingefügt werden können. Im Programmcode und in der Dokumentation sind die dazu notwendigen Methoden bereits angelegt und gekennzeichnet. Das Kapitel \ref{subsubsec:ErgSoftwErweiterung} gibt dazu detailliertere Hinweise.
Aufgaben wie die Selektion der zu betrachtenden Netze, die einfache Adaption von Parametern zur Simulation, die persistente Speicherung der Konfigurationen und Netzstrukturen und die graphische Ausgabe der errechneten Ergebnisse übernimmt der aktuelle Softwarestand bereits. Die Grundlage dafür bietet die Einbindung der frei verfügbaren Java-Bibliothek „jFreeChart“\footnote{„JFreeChart is a free 100\% Java chart library that makes it easy for developers to display professional quality charts in their applications. JFreeChart's extensive feature set includes, a consistent and well-documented API, supporting a wide range of chart types; a flexible design that is easy to extend,[…] support for many output types, including Swing and JavaFX components, image files (including PNG and JPEG).[…] JFreeChart is open source or, more specifically, free software. It is distributed under the terms of the GNU Lesser General Public Licence (LGPL), which permits use in proprietary applications.“ \cite{jdbc}} und des offiziellen JDBC-Treibers\footnote{Frei verfügbar unter https://dev.mysql.com/downloads/connector/j/5.1.html} zur Verbindung mit der MySQL-Datenbank.
\begin{figure}[ht]
	\centering
	\includegraphics[width=0.5\textwidth]{ErgSoftwareMVC}
	\caption{Schichtenarchitektur des EnergyNetSim (eigene Darstellung)}
	\label{fig:ErgSoftwareMVC}
\end{figure}
Das Zusammenspiel zwischen Java-Anwendung und der MySQL-Datenbank zeigt
Abbildung \ref{fig:ErgSoftwareMVC}. Bei der Entwicklung wurde ein „Model-View-Controller“ (kurz: MVC)-Entwurfsmuster gewählt, das die Trennung der Programmstruktur in die logischen Schichten „Datenhaltung“, „Programmlogik“ und „Präsentation“ ermöglicht.
Das Programm wird mitsamt den Datenbank-Skripten zur Erstellung der Datenbankstrukturen auf zwei Arten zur Verfügung gestellt:
\begin{itemize}
\item Als offenes Repository „EnergyNetSim/Simulator.git" über den Versionsverwaltungsdienst GitHub: https://github.com/EnergyNetSim/Simulator.git
\item Als ausführbares Java-Archiv (JAR), das in der Anlage zum vorliegenden Bericht enthalten ist.
Auf die notwendigen Schritte zur Installation der erstellten Software wird im nachfolgenden Kapitel eingegangen.
\end{itemize}

\subsubsection{Installation der Software}
Von den zwei vorgestellten möglichen Wegen, über die Simulationssoftware zu beziehen ist, wird im Anschluss die Inbetriebnahme mittels JAR-Datei beschrieben. Das Kapitel richtet sich somit vornehmlich an Endanwender.
Die Einbindung des GitHub-Repositories ist vor allem für Entwickler interessant und erfordert daher Kenntnisse im Umgang mit der Versionsverwaltungssoftware Git, einer zusätzlichen Java-Entwicklungsumgebung, beispielsweise IntelliJ oder Eclipse. Des Weiteren muss das Java-Development-Kit in seiner aktuellsten Version installiert werden. Was gerade in Kurzform skizziert wurde, würde in einer ausführlichen Fassung den Rahmen des Berichts sprengen. Außerdem existiert eine ausreichende Zahl an Online-Communities und Anleitungen\footnote{Auf der Webseite von GitHub findet sich eine Sammlung von Links zum Erlernen von Git: https://help.github.com/articles/good-resources-for-learning-git-and-github/
Ebenso bietet Oracle eine ausführliche Java-Dokumentation: http://www.oracle.com/technetwork/java/javase/downloads/index.html} zur Verwendung der beschriebenen Programme, sodass aus einer erneuten Schilderung kein wissenschaftlicher Mehrwehrt entstünde.
Bis das Java-Programm mittels JAR-Datei ausgeführt werden kann, sind die in Abbildung \ref{fig:ErgSoftwareInst1} abgebildeten Schritte erforderlich.
\begin{figure}[ht]
	\centering
	\includegraphics[width=0.8\textwidth]{ErgSoftwareInst1}
	\caption{Schritte der Installation}
	\label{fig:ErgSoftwareInst1}
\end{figure}
Voraussetzung ist das Vorhandensein der aktuellsten Version der Java-Laufzeitumgebung, kurz „JRE“. Innerhalb dieser Umgebung kann die Anwendung betriebssystemunabhängig in der Java-Virtual-Maschine (kurz: „JVM“) ausgeführt werden. Das Java Runtime Environment ist über die Webseite der Firma ORACLE\footnote{Zum Zeitpunkt der Abgabe des Berichtes ist der Download über folgenden Link möglich: http://www.oracle.com/technetwork/java/javase/downloads/jre8-downloads-2133155.html} zu beziehen und zu installieren.
Weil die Java-Anwendung auf eine MySQL-Datenbank als Persistenzschicht zugreifen soll, wird zusätzlich ein lokaler MySQL-Server benötigt, der über die IP-Adresse 127.0.0.1 / „localhost“ und den Port 3306 erreichbar ist. Das Entwicklerteam empfiehlt den Einsatz des Open-Source-Datenbankservers „MySQL Community Server“, der ebenfalls von ORACLE\footnote{Zum Zeitpunkt der Abgabe des Berichtes ist der Download über folgenden Link möglich: http://dev.mysql.com/downloads/mysql/} unter der GNU General Public License kostenlos angeboten wird. Ebenfalls empfehlenswert ist die Verwendung des Datenbank-Modellierungs- und Manipulationswerkzeugs „MySQL Workbench“, das über denselben Weg bezogen werden kann.
Nach der erfolgreichen Installation der beschriebenen Komponenten muss der SQL-Server gestartet werden, um die Datenbank anzulegen. Für Nutzer eines Windows-Betriebssystems sind dazu folgende Schritte erforderlich:

\begin{description}
\item [Starten des MySQL-Server-Dienstes:]
Durch Klick auf „starten“ wird der SQL-Server hochgefahren und der Status ändert sich in „Gestartet“.
\begin{figure}[ht]
	\centering
	\includegraphics[width=0.8\textwidth]{ErgSoftwareInst2}
	\caption{Starten des MySQL-Server-Dienstes}
	\label{fig:ErgSoftwareInst2}
\end{figure}
\item [Anlegen einer neuen Verbindung zum Server:]
Innerhalb der MySQL-Workbench wird eine Verbindung zum MySQL-Server mit den in Tabelle \ref{tab:mysql} aufgeführten Daten hergestellt.
\begin{table}[ht]
\centering
	 \begin{tabular}{|l|l|l|}
	 \hline
	Parameter & Wert\\
	\hline
	\hline
	Connection Method & Standard (TCP/IP)\\
	 \hline
	Hostname & 127.0.0.1\\
	 \hline
	Port & 3306\\
	 \hline
	Username & NetSimUser\\
	 \hline
	Passwort & NetSimUser\\
	 \hline
	 \end{tabular}
\caption{Parameter für das Einrichten des DB-Verbindung in der MySQL Workbench}
\label{tab:mysql}
\end{table}
\begin{figure}[ht]
	\centering
	\includegraphics[width=0.8\textwidth]{ErgSoftwareInst3}
	\caption{Konfigurieren einer neuen Verbindung zum MySQL-Server}
	\label{fig:ErgSoftwareInst3}
\end{figure}
\item [Importieren des Datenbank-Schemas:] Das Initialisierungsskript der Datenbank, welches sich im Anhang des Projektberichts angehängt ist, kann daraufhin über Server $ > $ Import Data $ > $ Import from Self-Contained File auf dem neuen MySQL-Server ausgeführt werden. Im Ergebnis existiert nun das Schema „energynetsimdb“.
\begin{figure}[ht]
	\centering
	\includegraphics[width=0.4\textwidth]{ErgSoftwareInst4}
	\caption{Das importierte Datenbankschema „energynetsimdb“}
	\label{fig:ErgSoftwareInst4}
\end{figure}
\item [Ausführen der JAR-Datei:] Durch Doppelklick auf die Datei „EnergyNetSim.jar“ wird das Programm gestartet. Im weiteren Verlauf wird die Verwendung des Programmes aus Endanwender-Sicht geschildert.
\end{description}

\subsubsection{Benutzung der Software zum aktuellem Stand}
Da momentan nur das Rahmenwerk für eine spätere Implementierung von Simulationsansätzen realisiert ist, bietet das Programm nur eine eingeschränkte Funktionalität für den Endnutzer. Auf Erweiterungsmöglichkeiten und dafür vorgesehene Strukturen im Kode wird im Anschluss an dieses Kapitel eingegangen.
\begin{figure}[ht]
	\centering
	\includegraphics[width=0.6\textwidth]{ErgSoftwareUse1}
	\caption{Hauptfenster nach dem Starten der Software}
	\label{fig:ErgSoftwareUse1}
\end{figure}
Sofern eine Verbindung mit der Datenbank aufgebaut werden konnte, erhält der Benutzer nach Start der Anwendung das Hauptfenster wie in Abbildung 6 angezeigt. Es besteht die Möglichkeit zur Auswahl verschiedener Netze, die in der Datenbank angelegt wurden, über den Menüpunkt „Network selection“ und zur Änderung von Parametern wie beispielsweise dem Strompreis oder dem Netzlastprofil über die Schaltfläche „Settings“.
\begin{figure}[ht]
	\centering
	\includegraphics[width=0.2\textwidth]{ErgSoftwareUse2}
	\caption{Dialoge „Select networks“ und „Settings“}
	\label{fig:ErgSoftwareUse2}
\end{figure}
\begin{figure}[ht]
	\centering
	\includegraphics[width=0.4\textwidth]{ErgSoftwareUse3}
	\caption{Dialoge „Select networks“ und „Settings“}
	\label{fig:ErgSoftwareUse3}
\end{figure}
Die geänderten Werte werden in der Datenbank hinterlegt, momentan jedoch von der Kalkulationsmethode nicht weiterverwendet. Stattdessen sind in ihr Werte für den Stromverbrauch, die Netzlast und Kosten hinterlegt, deren Ermittlung in Kapitel \ref{subsec:MethSch} beschrieben wird.
\begin{figure}[ht]
	\centering
	\includegraphics[width=0.8\textwidth]{ErgSoftwareUse4}
	\caption{Ausgabe der Ergebnisse in Diagrammform}
	\label{fig:ErgSoftwareUse4}
\end{figure}
Durch Klick auf die Schaltfläche „Calculate“ werden die Werte in Form von Histogrammen visualisiert.

\subsubsection{Erweiterungsmöglichkeiten} \label{subsubsec:ErgSoftwErweiterung}
Die beschriebene Software stellt freilich keine abgeschlossene Lösung zur Wirtschaftlichkeitsbetrachtung von energieeffizienten Konzepten in simulierten Netzen dar. Wie schon zu Beginn des Kapitels erwähnt, konnte die Entwicklung nicht in der zur Verfügung gestellten Zeit abgeschlossen werden. Vielmehr war die Zielsetzung der Arbeit, eine Anwendung zu erstellen, die es zukünftigen tiefergreifenden Forschungsvorhaben ermöglicht, die fehlenden Algorithmen zu implementieren, welche in Kapitel \ref{subsec:ErgAlg} beschrieben sind, ohne auf Aspekte der grafischen Ausgabe sowie der Anbindung an eine externe Datenbank achten zu müssen.
Das vorliegende Java-Programm erfüllt diese Anforderungen, indem es folgende vordefinierte Methoden und Datenbankstrukturen enthält, die eine Erweiterung um Simulationsalgorithmen zulassen.

\begin{description}
\item [Die Funktion „calculate()“ in der Klasse „MainModel“.] Bei Klick auf die Schaltfläche „Calculate“ wird über den Controller in der Classe „MainModel“ die Funktion „public void calculate()“ aufgerufen. Von dort können der komplette Simulationsalgorithmus gestartet, neue Instanzen von anderen Model-Classen erzeugt und die erhaltenen Ergebnisse in Form einer Liste gespeichert werden.
\item [Das Package „models“.] Das Java-Package „models“ bietet Platz für weitere Klassendefinitionen, die von der „calculate()“-Methode der Klasse „MainModel“ aufgerufen werden und ihrerseits auf die MySQL-Datenbank über die vorhandene Klasse „DatabaseQueries“ zugreifen können. Beispielhaft wurden für die Gerätehardware und die physisch vorhandenen Verbindungen zwischen Knoten die Model-Klassen „HardwareDevices“ und „Link“ angelegt, in denen zukünftig Programmlogik implementiert werden kann.
\item [Das Datenbankschema „energynetsimdb“.] Im Rahmen des Software-Engineering-Prozesses wurde ein Datenbankschema entwickelt, das die Datenstruktur für spätere Simulationsalgorithmen abbilden kann. Das zugehörige Entity-Relationship-Diagramm wurde bereits in Kapitel \ref{subsec:MethSoftwareEng} vorgestellt und ermöglicht die Generierung der SQL-Create-Befehle für die noch nicht angelegten Relationen.
\end{description}

Zusammenfassend kann festgestellt werden, dass lediglich Änderungen in der Daten- und Logikschicht innerhalb der bereits vorgegebenen Strukturen durchgeführt werden müssen, um den noch nicht kodierten Simulationsalgorithmus einzubinden. Dazu liegt die vollständige Dokumentation der Software dem Anhang bei.


\subsection{Diskussion und Erkenntnisse} \label{subsec:ErgDiskussion}
%TODO Veronika
Zu Beginn dieser Arbeit setzte sich die Projektgruppe zum Ziel, eine sachlich möglichst richtige Simulation der Netzlast zu entwickeln, um den Energieverbrauch der beiden Netze so realistisch wie möglich abschätzen zu können. Auf diese Art und Weise sollte ein valides Ergebnis entstehen.

Schon während der Literaturrecherche war überraschend, dass vorhandene Arbeiten auf dem Gebiet sich immer mit einem kleinen Untergebiet des Themas beschäftigen -- die eine große Überblicksarbeit zum Stand der Forschung unter Kombination aller vorhandenen Technologien gab es nicht.

Sobald dann mit der Entwicklung des Algorithmus angefangen worden war, stellte sich heraus, dass für ein valides Ergebnis die Simulation sehr komplex werden musste. Hierbei stellten sich folgende Herausforderungen:
\begin{itemize}
	\item In den Nodes sind jeweils Geräte verschiedener Layer nötig, um das weitere Routing zu gewährleisten. 
	\item Für eine realistische Berechnung der Netzlast reicht eine stundenweise Iteration nicht aus. Diese Annahme bedeutet, dass von jeder Quelle pro Stunde ein konstanter Datenstrom ausgeht, was in der Realität so nicht der Fall ist.
	\item Die Einhaltung der QoS Parameter konnten mit dem gewählten Ansatz nicht ge\-währ\-lei\-stet werden.
	\item Es ist unklar, welche Konzepte zur Erhöhung der Energieeffizienz miteinander kompatibel sind.
\end{itemize}

Um eine Umsetzung des Algorithmus zu ermöglichen, mussten also sehr viele Annahmen getroffen werden. Hierdurch geht das eigentliche Ziel, eine sachlich richtige Simulation durchzuführen, weitgehend verloren, da im Vergleich zu echten existierenden Backbone-Netzwerken die Vergleichbarkeit und damit die Praxistauglichkeit der Simulation nicht gegeben ist.

Es zeigt sich also, dass der Algorithmus in der gegebenen Projektdauer dieser Arbeit nicht in der gewünschten Qualität zu implementieren ist. Nach Rücksprache mit dem Betreuer dieser Arbeit wurde beschlossen, den Algorithmus in der geplanten Form zu dokumentieren (vgl. Kapitel \ref{sec:Erg}) sowie das Grundgerüst der Softwarelösung zu implementieren. 

%MISSING Numerisches Ergebnis der Abschätzung diskutieren. Unterschied deutlich?
%TODO 
Bei Betrachtung des Gesamtverbrauchs resultiert ein Energieeinsparpotenzial von 23,5 \% pro Jahr  \cite[5]{Chiaraviglio2009}, in Geldeinheiten\cite{Proteus2016} bewertet, 91.500 €. 
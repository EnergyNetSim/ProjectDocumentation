\section{Vorgehen}

\subsection{Modellierung}
Zur Betrachtung der Wirtschaftlichkeit eines energieeffizienten Netzes muss ein Vergleich zwischen einem Standardnetz (Business-As-Usual-Netz) und einem energieeffizienten Netz durchgeführt werden. Bei der Spezifikation des Netzes liegt ein Hauptaugenmerk auf der Eingrenzung des Netzes. Im Rahmen dieser Arbeit soll dabei nur das Core-Netz betrachtet werden. Der eingerahmte Bereich aus Abbildung X beinhaltet die „Broadband Network Gateway“(BNG), „Label-Switch-Router“ (LSR) und „Label-Edge-Router“ (LER).


Bild netzwerk mi Eingrenzung





%TODO Carmen: Bitte noch die folgende Abbildung miteinbeziehen (Netzlastprofil wurde ja ebenfalls modelliert)



Damit ein Vergleich möglich ist, müssen beide Netze identisch modelliert werden. Das statische soll in der Simulation nicht  verändert werden. Das zweite Netz soll die im Kapitel „Stand der Forschung“ beschriebenen Technologien anwenden und die Effizienz des Netzes damit zu steigern. Um sowohl das Standard als auch das effiziente Netz möglichst realistisch zu modellieren, sollen die Geräte pro Layer gewählt werden. Die Geräte werden dann in der Datenbank konfiguriert. Die Modellierung des Netzes erfolgt durch die Erstellung von Verbindungen zwischen den Geräten. Zuvor muss eine Anzahl von Knoten im Netz definiert werden. 
\begin{figure}[!ht]
	\centering
	\includegraphics[width=0.8\textwidth]{VorgModLast}
	\caption{Idealtypische Netzlast-Kurve (nach \cite[3]{Chiaraviglio2009})} 
	\label{fig:VorgModLast}
\end{figure}


\subsection{Entwicklung des Routing-Algorithmus}\label{subsec:MethAlg}
Zur Simulation des Netzwerkes, der Lastverteilung und darauf basierend zur Berechnung des elektrischen Stromverbrauches des Netzwerkes ist ein iteratives Berechnen für einzelne Zeitabschnitte abhängig von der Netzlast, des Traffic-Ursprungs (Quelle) und des Traffic-Ziels(Senke) notwendig.
Die iterative Berechnung erlaubt die Bestimmung des Stromverbrauches für einen bestimmten Zeitraum. Dies ermöglicht die Beachtung der Uhrzeit abhängigen Belastung der Carrier-Netze und darauf basierend das An- oder Ausschalten von Netzkomponenten zur Steigerung der Energieeffizienz des Carrier-Netzes.  


\begin{figure}[!ht]
	\centering
	\includegraphics[width=1\textwidth]{Netzrouting}
	\caption{Modellnetz mit mehreren Routingmöglichkeiten}
	\label{fig:Netzrouting}
\end{figure}


Es stehen zur Berechnung des Stromverbrauchs je Zeitabschnitt generell mehrere Ansätze zur Verfügung. Der logisch einfachste Ansatz wäre je Uhrzeit von einer bestimmten zuvor als Datenbankeintrag festgelegten Nutzungsauslastung je Verbindung auszugehen (feste Werte bezüglich Auslastung je Gerät). Dies würde ein einfaches Bestimmen der einzelnen Geräteauslastungen und anschließendem addieren der Werte zu einem Gesamtstromverbrauch pro Zeiteinheit ermöglichen. Diese Ergebnisse könnten dann als Tabelle und Histogramm dargestellt werden.


Dieser Lösungsweg setzt allerdings eine sehr detaillierte und statische Modellierung eines Beispielnetzes und allen dazugehörigen Auslastungs- und Verbrauchswerten je Zeitabschnitt voraus. Aufgrund dieses Konfigurationsaufwands wird ein Anwender folglich einen großen Zeitraum je Iteration auswählen, was zu einer sehr groben und realitätsfernen Simulation führen würde. Energiespartechniken wie das dynamische Abschalten von einzelnen Verbindungen und umrouten der Netzlast auf andere noch nicht optimal ausgelastete Verbindungen muss der Anwender in diesem Verfahren selbstständig je Iteration umsetzen.


Als weitere Möglichkeit zur Berechnung der optimalen Stromverbrauchswerte bietet sich das automatisierte, in einem Algorithmus implementierte Berechnen der einzelnen Netzlast-Routing-Möglichkeiten und anschließend der Wahl des nach zuvor definiertem Regelsatz am besten geeigneten Routing je Netzlast-Element an.

Ein solcher Algorithmus bietet den Vorteil, dass der Anwender neben dem Netzaufbau nur die Netzlast mit Iterationszeitraum, Quelle, Senke und Menge spezifizieren muss und die Software aus diesen Daten selbstständig eine für das gesamte Problem  gute Lösung bestehend aus den einzelnen Routing-Entscheidungen errechnet.


Dieser Greedy-Algorithmus liefert nicht für jeden Anwendungsfall die beste Lösung, da die jeweilige Entscheidung nur bei Betrachtung des einzelnen Netzlast-Elements erfolgt, anstelle alle Routingmöglichkeiten für alle Netzlastelemente einer Iteration zu betrachten, ist dafür allerdings hinsichtlich Zeitkomplexität für Netze mit vielen Hardwarekomponenten deutlich effizienter.
Ein weiterer Nachteil eines solchen Algorithmus ist, dass dieser nicht die im Netz des Anwenders verwendeten Regeln zum Routing nachahmt, sondern das Netz als teilvermascht mit Gleichberechtigung der einzelnen Strecken bei Anwendung des fest implementierten Regelsatzes betrachtet.


Die Implementierung eines Algorithmus, der alle Spezialitäten des Netzes des Anwenders beachtet, würde den Rahmen dieser Arbeit wesentlich überschreiten. Eine evolutionäre Erweiterung des zuvor beschriebenen Algorithmus um durch den Anwender zu spezifizierende Regelsätze ist als Weiterentwicklung dieser Arbeit allerdings durchaus möglich.


\subsection{Softwareentwicklung}\label{subsec:MethSoftwareEng}
Die zuvor beschriebene Modellierung zur Kalkulation und Simulation von energieeffizierten Netzen soll im nächsten Schritt softwaretechnisch umgesetzt werden. Als Vorgehensmodell hat sich das Projektteam an dem Wasserfallmodell mit den folgenden Entwicklungsstufen orientiert:
\begin{itemize}
\item Anforderungsanalyse
\item Spezifikation
\item Systemdesign
\item Programmierung / Implementierung
\item Test
\end{itemize}
Die Analyse der Anforderungen an die Simulationssoftware erfordert eine zielgerichtete Recherche über bereits vorhandene Technologien und Methoden. Ebenfalls müssen die Funktionalitäten, die das Simulationstool bereitstellen soll, abgegrenzt werden. Die Anforderungen werden zunächst als Ideen und Möglichkeiten gesammelt und dann in Form von Anforderungsspezifikationen definiert. 


Das System soll ein statisches Netz mit einer festgelegten Anzahl an Verbindungen abbilden. Alle Daten zu Geräten, Netzen, Verbindungen und Netzprofilen sollen in einer MySQL-Datenbank gespeichert werden. Die Verwendeten Geräte sowie neue Netzprofile sollen über die Datenbank eingepflegt werden können. Für das Netzprofil sollen Traffic-Daten (Netzprofil, Last, Quelle, Senke, Uhrzeit) verwendet werden. Aus den Daten entsteht eine gleichbleibende Auslastungskurve, anhand dessen die Verbindungen für die jeweilige Uhrzeit definiert werden können. Um das Netz möglichst realistisch zu simulieren bekommen die einzelnen Verbindungen eine Länge zugeordnet. Anhand dieser Länge (mehr als 80 km) wird definiert, ob ein Verstärker hinzugefügt werden muss, um die Verbindung zu verbessern. Das definierte, statische Netz soll die Verwendung verschiedener Technologien und Methoden zur Senkung des Energieverbrauchs möglich machen. Zur Abschaltung von Switches und Ports soll ein Algorithmus entwickelt werden, der auf dem „Exhaustive Greedy Algorithm“(QUELLE!!!) aufbaut.
Im Zuge der Anforderungsanalyse und der Spezifikation der Software wurden die folgenden funktionalen, nicht funktionalen und datenbezogenen Anforderungen erstellt:



Funktionale Anforderungen:
\begin{itemize}
	\item A1: Über die Datenbank kann ein Netz, bestehend aus Knoten (Geräte) und Kanten (Verbindungen) modelliert werden
	\item A2: Das Anlegen neuer Geräte und Geräte\-konfi\-gura\-tionen ist über die Datenbank möglich
	\item A3: Das Verändern von Geräteparametern ist über die Datenbank möglich
	\item A3a: Das Austauschen von Gerätetypen ist über die Datenbank möglich
	\item A3b: Das Austauschen von Gerätetypen pro Schicht ist über eine GUI möglich                       
	\item A4: Das System berechnet anhand des modellierten Netzes, der eingestellten		 Gerätekonfigurationen, der gewählten Energiespartechnologien und des vorliegenden Lastprofils den Stromverbrauch.
	\item A5: Das System berechnet die Stromkosten anhand des Stromverbrauches und einem vorgegebenen Umrechnungsfaktor 
	\item A6: Zur Vereinfachung von Berechnungen müssen Annahmen getroffen werden, die in der DB hinterlegt werden
	\item A7: Es werden generische Geräte basierend auf dem Dokument "<Dokumentname einfügen>" verwendet
	\item A8: Zur Abschaltung von Switches/Ports dient der "Exhaustive Greedy Algorithm"
	\item A9: Der simulierte Stromverbrauch im Laufe eines Tages soll pro Stunde graphisch ausgegeben werden
	\item A10: Die Lastkurve wird graphisch dargestellt
\end{itemize}


Nicht funktionale Anforderungen:
\begin{itemize}
	\item NFA1: Das System soll als Client-only Applikation vorliegen
	\item NFA2: Das System verwendet eine MySQL-Datenbank als Persistenzschicht
\end{itemize}


Datenbezogene Anforderungen:
\begin{itemize}
	\item DA1: Verbindungen wird eine Länge zugewiesen
	\item DA2: Für Gerätetypen können relevante Leistungs- und Verbrauchsparameter hinterlegt werden
	\item DA3: Ein Netz entsteht durch das Verbinden von Geräteinstanzen und der Zuordnung zu einer NetzID
	\item DA4: Annahmen werden durch einen eindeutigen Namen und einen Wert abgebildet
	\item DA5:  Die Netzlast wird in Form von Datenpaketen mit den Attributen Volumen, Quelle, Senke, Uhrzeit, Profil-ID definiert
\end{itemize}


In der Systemdesign-Phase werden einige Entscheidungen bezüglich der Softwareentwicklung getroffen. Als Programmiersprache für das Projekt wurde objektorientiertes Java ausgewählt. Die bereits vorhandene Entwicklungserfahrung im Team mit dieser Sprache und der sehr ausgeprägten Verfügbarkeit von kostenfreien Entwicklungstools und fertigen Bibliotheken waren für die Entscheidung ausschlaggebend.
Die Architektur besteht aus einer 2-Tier Kombination aus einer relationalen MySQL/MariaDB Datenbank und einer Fat-Client Anwendung. Die Datenbank wird dabei zur Speicherung der zur Erstellung eines Netzes notwendigen Daten, sowie Hardware-, Topologie- und Netzwerkkonfiguration verwendet. Die Anwendung hingegen wird zur Parameterkonfiguration, Berechnung der Zielwerte und Ausgabe der Resultate verwendet.


Die Vorteile dieser Architektur sind die gute Portabilität der resultierenden Lösung und die Ermöglichung einer einfachen und schnellen Entwicklung. Es müssen keine weiteren Anwendungsserverkomponenten je Entwicklungsumgebung installiert und konfiguriert werden. Die SQL-Datenbank verwendet eine übliche Datenbank-Standardsoftware und kann per SQL-Script vollautomatisch konfiguriert werden.
Das Datenbank-Konzept ist entscheidend für die Softwareentwicklung. Es werden alle relevanten Daten für Berechnungen, Konfigurationen und Darstellungen gespeichert. Auf der Basis der Datenbankstruktur kann die Entwicklung der Softwarelösung weitergeführt werden.


\begin{figure}[!ht]
	\centering
	\includegraphics[width=\textwidth]{MethSoftwareER}
	\caption{ER-Diagramm} %TODO Bezeichnung anpassen
	\label{fig:MethSoftwareER}
\end{figure}


%----> Erklärung ER-Diagramm <----
Zur Vorbereitung der Programmierung wurde der Algorithmus für das Routing in der Systemdesign-Phase entwickelt.  Dieser Algorithmus wird in den  Kapiteln \ref{subsec:MethAlg} \& \ref{subsec:ErgAlg} detailliert erklärt.


In der Programmierungsphase wurde ein Grundgerüst für die weitere Entwicklung der Software geschaffen. Die Datenbank wurde auf der Basis des zuvor erstellten Konzeptes eingebunden. Im Rahmen der Weiterentwicklung kann der Routing-Algorithmus und die definierten Anforderungen implementiert werden. Der Algorithmus koordiniert das Routing des Traffics durch das definierte Netz. Mit der Hilfe von weiteren Annahmen und Algorithmen kann die Software zu einem System entwickelt werden, das verschiedene Technologien und Methoden zur Energiereduzierung einsetzt. Das Programm könnte die verschiedenen Methoden möglichst realitätsgetreu anwenden und die dadurch gesparte Energie graphisch darstellen. Über die GUI könnte eine Konfiguration möglich sein.  Darüber könnten verschiedene Technologien aktiviert und Geräte ausgetauscht werden. 


Die beschriebene Software konnte innerhalb dieses Projektes nicht fertigstellt werden. Gründe dafür waren die Begrenzung der Bearbeitungszeit auf drei Monate und vermehrt auftretende komplexe Problemstellungen. Die einzelnen Faktoren der Komplexität werden in Kapitel 4.4 erläutert.


Eine ausgereifte Testphase ist erst nach der Fertigstellung der Software sinnvoll. Die implementierten Module und Teilfunktionalitäten werden vom jeweiligen Entwickler direkt getestet. Ein abschließender Test der bis zum Ende der Bearbeitungszeit programmierten Funktionalitäten, wird durchgeführt.


%TODO wohin mit dem folgenden Abschnitt?
Um den Energieverbrauch im Carrier-Netzwerk berechnen zu können, bedarf es an Werten der im Backbone verwendeten Netzkomponenten. Innerhalb der Simulation wird für die Berechnung des Gesamtverbrauchs auf die definierten Werte zurückgegriffen, die in einer Datenbank gespeichert sind. Ziel dieser Arbeit ist die Simulation und Berechnung des gesamtheitlichen Energieverbrauchs. Deshalb verwendet die Berechnung die von den beiden Wissenschaftler Ward Van Heddeghem und Filip Idzikowski in ihrer Veröffentlichung \cite{vanhedde} zusammengetragenen Werte. Die Quelle beinhaltet zum einen das analytische Model der Berechnung und zum anderen ein Datenblatt \cite{vanhsheet} des Energieverbrauchs der unterschiedlichen Hersteller. Das Datenblatt gliedert die Geräte nach den Unterschiedlichen Layer IP/MPLS, Ethernet, OTN, WDM - (OSI-Layer: 3-2-1-1). Zu beachten ist beim Verwenden der Werte, dass es sich um Werte unter typischen Lastbedingungen handelt, die sich nach der Kapazität der Komponente richtet und nicht nach dem aktuellen Durchsatz. Des Weiteren geben die Werte nur dein Stromverbrauch für den Betrieb an, ein Verbrauch für Kühlung o.Ä. ist nicht enthalten.
Der Gesamtverbrauch des Core-Networks ergibt sich aus der Summe alle Verbrauche der einzelnen Schichten.



\subsection{Komplexität bei der Entwicklung}
Die Komplexität, die Unterschätzung der Einflussfaktoren und deren Beziehung zueinander werden häufig zu Problemen der Softwareentwicklung. In diesem Projekt konnte die Komplexität der Umsetzung der Aufgabenstellung erst in der Realisierungs- / Programmierungsphase erkannt werden. Im Zuge der Anforderungsanalyse und Spezifikation wurden Anforderungen definiert, die ein automatisiertes und konfigurierbares System zur Berechnung und Darstellung von energieeffizienten Netzen beschreiben. Während der Umsetzung der Anforderungen wurden komplexe Problemstellungen entdeckt, die erfasst und nach der Auswirkung auf das Projekt bewertet wurden. Bei dem Versuch diese Probleme zu lösen oder zu umgehen, sind weitere komplexe Schwierigkeiten aufgetaucht. Aufgrund von zwei Faktoren wurde die Entwicklung der Software bei einem Grundgerüst belassen. Ein Mangel an Ressourcen machte es nicht möglich, die auftauchenden komplexen Problemstellungen genau im System abzubilden. Daher mussten Annahmen getroffen werden, die die Genauigkeit der vom System erwarteten Ergebnisse stark reduzierte. Da ungenaue Ergebnisse in diesem Forschungsgebiet nicht zielführend und akzeptabel sind, wurde die Software nicht entsprechend der zuvor definierten Anforderungen weiter entwickelt. Die Komplexität der Entwicklung soll im Folgenden detailliert beschrieben werden.  


Die Implementierung eines Netzes bedarf einem Lastprofil, mehreren Geräten und der jeweiligen Verbindung zwischen zwei Geräten. Der in Kapitel 4.3 beschriebene Algorithmus soll das Routing durch das Netz pro Zeitintervall berechnen. Ein Zeitintervall soll unterschiedlich einstellbar sein. Es muss sichergestellt werden, dass keine Verbindungen überlastet sind. Dafür muss der Algorithmus zunächst die optimale Route berechnen und diese dann mit allen bereits definierten Verbindungen vergleichen. Wurde die gleiche Verbindung bereits gewählt muss geprüft werden, ob die Geräte der Verbindung voll ausgelastet sind. Ist die maximale Auslastung erreicht, muss entschieden werden, welche Verbindung bestehen bleibt und für welche Verbindung eine alternative Route gefunden werden muss. Dies kann nur über eine komplizierte Berechnung und einen Vergleich durch den Algorithmus durchgeführt werden. Ein weiteres Problem beim Routing ist die Blockierung der Ports. In der Realität werden Ports nur eine bestimmt Zeit belastet. Diese Zeit ist abhängig von der Datenmenge und Geschwindigkeit des Ports. Die Berechnung der Zeit, die ein Port von einem Datenpaket blockiert wird, macht  die Verwendung von festen Zeitintervallen fast unmöglich. Es müsste  definiert werden, wie lange der Port mit welchem Volumen belegt ist. Dadurch würden die Zeitschlitze zur kleinsten Einheit. Um diese Komplexität zu minimieren wurde die Annahme getroffen, dass ein Datenpaket einen Port für die gesamte Zeiteinheit belegt. 


Eine weitere Annahme musste bei der Berechnung des Stromverbrauchs getroffen werden. Jedes Gerät besitzt einen vom Hersteller vorgegebenen Stromverbrauch pro Port. Der Stromverbrauch für das gesamte Gerät ergibt sich nicht aus der Multiplikation von Stromverbrauch pro Port und der Anzahl an Ports. Der Grund dafür ist, dass das Gerät auch Strom verbraucht, wenn kein Datenpaket über die Leitung geht. Daher ist der genaue Stromverbrauch bei heruntergefahrenen Ports nur ungefähr abschätzbar. Des Weiteren  muss der Stromverbrauch für Geräte, die sich im Energiesparmodus befinden, geschätzt werden. Zu dieser Problematik kommt noch die Komplexität der Berechnung des Stromverbrauchs des gesamten Netzes dazu. Mit heruntergefahrenen und in der Geschwindigkeit reduzierten Ports und Geräten, muss zu jedem Zeitpunkt der Status des jeweiligen Ports und Geräts bekannt sein.  


\subsection{Abschätzung des Energieverbrauchs} \label{subsec:MethSch}

Obgleich die Komplexität des vollständigen Simulationsalgorithmus eine Implementierung im Rahmen der Arbeit verhinderte, wurde dennoch ein beispielhafter Vergleich des Energieverbrauchs und damit der Stromkosten eines herkömmlichen und eines energieeffizienten Telekommunikationsnetzwerkes durchgeführt, dessen Ergebnisse in der aktuellen Software-Version statisch hinterlegt sind. Als Grundlage wurde die Fallstudie eines italienischen Forscher-Konsortiums aus Turin herangezogen, die die Energieeinsparungspotenziale im realen Netz eines der größten italienischen Internet-Service-Provider analysiert\cite{Chiaraviglio2009}. Das untersuchte Netzwerk gliedert sich in die vier Ebenen "Core", "Backbone", "Metro" und "Feeder", deren Energieverbrauch und Anzahl der Netzelemente in der Tabelle \ref{tab:beispielnetz} aufgelistet sind\cite[2]{Chiaraviglio2009}. Die Energie zur Kühlung der Netzelemente wird in der Fallstudie nicht betrachtet.




\begin{table}[ht]
\centering
\begin{tabularx}{\textwidth}{ | r | l | X | X | }
	\hline
	Knotentyp & Anzahl Knoten & Energieverbrauch \newline je Knoten [kWh] & Energieverbrauch \newline je Ebene [kWh]\\ \hline\hline
Core & 8 & 10 & 80\\ \hline
Backbone & 52 & 3 & 156\\ \hline
Metro & 52 & 1 & 52\\ \hline
Feeder & 260 & 2 & 520\\ \hline
\end{tabularx}
\caption{Aufbau des Beispielnetzes}
\label{tab:beispielnetz}
\end{table}

Somit ergibt sich der Energieverbrauch pro Stunde für das nicht-optimierte Beispielnetz als Summe aus den Einzelverbräuchen der Netzebenen und den Verbindungen mit optischen Verstärkern.

 \begin{equation}
P_{ges/h} = P_{Node} + P_{Link} = 0,808 MWh + 0,594 MWh = 1,42 MWh
 \end{equation}


Als normalisierte Lastkurve wird die in \ref{fig:VorgModLast} dargestellte Sinus-Funktion sowohl für das her\-kömm\-liche als auch das energieeffiziente Netzwerk angenommen und im Java-Programm abgebildet. In der Fallstudie wurde die Energieffizienz durch dynamisches Abschalten von Geräten und Verbindungen, die keiner Netzwerklast ausgesetzt sind, erhöht.\cite[3]{Chiaraviglio2009} Für das nicht-optimierte Netz wird aufgrund fehlender Detaildaten angenommen, dass der Energieverbrauch unabhängig von der Netzlast ist. Dabei ergeben sich die in Tabelle \ref{tab:stromverbrauch} angegebenen Stromverbrauchswerte pro Stunde für beide Netze, welche in der Simulationssoftware statisch hinterlegt wurden und als Grundlage der Diagrammerstellung dienen.

\begin{table}[!ht]
\centering
\begin{tabular}{cccc}	
	\textbf{Zeit} & \multicolumn{2}{c}{\textbf{Stromverbrauch/Stunde} [MWh]}   \\
	& Herkömmliches Netz & Energieeffizientes Netz \\ \hline \hline
	0 Uhr & 1,42 & 0,95 \\ \hline
	1 Uhr & 1,42 & 0,95 \\ \hline
	2 Uhr & 1,42 & 0,95 \\ \hline
	3 Uhr & 1,42 & 0,95 \\ \hline
	4 Uhr & 1,42 & 0,95 \\ \hline
	5 Uhr & 1,42 & 0,95 \\ \hline
	6 Uhr & 1,42 & 0,95 \\ \hline
	7 Uhr & 1,42 & 0,95 \\ \hline
	8 Uhr & 1,42 & 1,08 \\ \hline
	9 Uhr & 1,42 & 1,13 \\ \hline
	10 Uhr & 1,42 & 1,15 \\ \hline
	11 Uhr & 1,42 & 1,18 \\ \hline
	12 Uhr & 1,42 & 1,21 \\ \hline
	13 Uhr & 1,42 & 1,24 \\ \hline
	14 Uhr & 1,42 & 1,27 \\ \hline
	15 Uhr & 1,42 & 1,27 \\ \hline
	16 Uhr & 1,42 & 1,24 \\ \hline
	17 Uhr & 1,42 & 1,20 \\ \hline
	18 Uhr & 1,42 & 1,16 \\ \hline
	19 Uhr & 1,42 & 1,14 \\ \hline
	20 Uhr & 1,42 & 1,12 \\ \hline
	21 Uhr & 1,42 & 0,95 \\ \hline
	22 Uhr & 1,42 & 0,95 \\ \hline
	23 Uhr & 1,42 & 0,95 \\ \hline\hline
	\textbf{SUMME} & \textbf{34} & \textbf{26}\\ 
\end{tabular}
\caption{Vergleich des Energieverbrauchs pro Stunde zwischen herkömmlichem und energieeffizientem Netz}
\label{tab:stromverbrauch}
\end{table}
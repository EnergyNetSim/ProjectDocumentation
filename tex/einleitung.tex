\section{Einleitung}
%Die Welt\-bevölkerung wächst täglich um circa 229.000 Menschen \cite{statista:zuwachs}. Die Strompreise in Deutschland werden bis 2025 um 70 Prozent steigen\cite{welt}. Die EU fordert 20 Prozent mehr Energieeffizienz bis 2020\cite{bmwi:energiepolitik}. Diese Aussagen zeigen, dass die wachsende Bevölkerung, der Klimawandel und die Verknappung der Energieressourcen in einem starken Zielkonflikt stehen. 
%Einen großen Anteil am Gesamtstromverbrauch haben die Telekommunikations- und Informationsindustrie. 2012 verbrauchten die herkömmlichen Rechenzentren in Deutschland über zehn Terawattstunden Strom, was etwa der Leistung von zwei Kernkraftwerken des Typs Brunsbüttel entspricht. Laut einer Prognose vom Fraunhofer Institut werden Informations- und Kommunikationstechnologien im Jahr 2020 zwölf Prozent des Gesamtstrombedarfs verbrauchen\cite{fraunhofer:energiebedarf}. Das ist damit zu begründen, dass zunehmend immer mehr Datenverkehr auf den Kommunikationsnetzen lasten und sich die Breitbandnachfrage erhöhen wird. 

%Die rasant steigende Nachfrage nach Bandbreite entsteht durch die Konvergenz von Medien- und Telekommunikationstechnologie. Orts- und zeitunabhängige TV-Nutzung, Video on Demand, Pay TV und HD sind beispielsweise treibende Faktoren für den Bedarf an Bandbreite. 
%Bedingt durch diese Entwicklungen gewinnen Energieverbrauch und Energieeffizienz von Kommunikationssystemen und -netzen immer mehr an Bedeutung. Denn \textquote{Green IT} kann dieser Entwicklung gegensteuern. Im Rahmen des Hochschulmoduls \textquote{Wissenschaftlich angeleitete Berufspraxis 3 (Technik)} sollte daher eine Simulationssoftware zur Dimensionierung des Energiebedarfs für ein fiktives effizientes Netz anhand von aktuellen Übertragungstechnologien entwickelt werden. 

%In diesem Bericht gilt es, die Forschungsergebnisse und wichtigen Erkenntnisse dieser Arbeit zu dokumentieren. Dazu soll zunächst die Problemstellung detailliert beschrieben und die Ziele der Simulationssoftware erklärt werden. Im darauf folgenden Kapitel wird der Stand der Forschung ausführlich betrachtet und im Rahmen dessen auf Carrier-Netzwerke, Möglichkeiten der Energieeinsparung und energieeffiziente Technologien eingegangen. Für weitere Forschungsprojekte in diesem Themengebiet ist insbesondere das Vorgehen der Projektgruppe von Interesse. Wesentliche Inhalte sind daher die Modellierung der Netze, die Softwareentwicklung, die Erstellung eines geeigneten Routingalgorithmus und dabei auftretende Komplexitäten. Die Ergebnisse der Projektarbeit werden darauf folgend im Zuge der Auswertung dokumentiert und beschrieben. Aufbauend darauf werden im Ausblick weitere zu erforschende Problemstellungen aufgezeigt. 



In den vergangenen Jahren gewannen Energieverbrauch und Energieeffizienz von Kommunikationssystemen und -netzen immer mehr an Bedeutung. Ein Grund dafür ist, dass Telekommunikations- und Informationssysteme mit einem großen Anteil am Gesamtenergieverbrauch beteiligt sind \cite{fraunhofer:energiebedarf}. Zukünftig werden immer mehr Geräte über Telekommunikationsnetze miteinander verbunden sein und dadurch mehr Netzwerkverkehr erzeugen \cite{cisco}.  Außerdem steigt die Nachfrage nach Bandbreite, die durch die Konvergenz von Medien- und Telekommunikationstechnologien bedingt ist \cite{anga}.  Orts- und zeitunabhängige TV-Nutzung, Video on Demand, Pay TV und HD sind beispielsweise treibende Faktoren für den Bedarf an Bandbreite. Aus diesen prognostizierten Entwicklungen ist abzuleiten, dass sich der Anteil der IKT-Branche am Gesamtenergieverbrauch nicht reduzieren wird, sofern die Telekommunikationstechnologien nicht effizienter gestaltet werden. In der Forschung existieren bereits diverse Ansätze zur Energieeinsparung während des Betriebes eines Netzes.


Die vorliegende Arbeit soll die Wirtschaftlichkeit von energieeffizienten Netzkonzepten untersuchen und dabei den Fokus auf den energieeffizienten Betrieb legen. Um den Vergleich zwischen energieeffizienten und traditionellen Netzkonzepten zu ermöglichen, wird als Ansatz die Entwicklung einer Simulationssoftware zur Ermittlung des operativen Energiebedarfs für  selbst zu definierende Netze und implementierte effizienzsteigernde Algorithmen gewählt.


Die erlangten Forschungsergebnisse dieses Projekts dokumentieren die folgenden Kapitel. Dazu sollen zunächst die Problemstellung detailliert beschrieben und die konkreten Ziele der Forschungsarbeit erklärt werden. Im darauf folgenden Kapitel wird der Stand der Forschung ausführlich betrachtet und im Rahmen dessen auf Carrier-Netzwerke, Möglichkeiten der Energieeinsparung und energieeffiziente Technologien eingegangen. Für weitere Forschungsprojekte in diesem Themengebiet ist insbesondere das Vorgehen der Projektgruppe interessant. Wesentliche Inhalte sind daher die Netzmodellierung, Erstellung eines geeigneten Routingalgorithmus, Softwareentwicklung und die dabei auftretende Komplexität. Es folgt die Abschätzung des Energieeinsparpotenzials eines realen Netzes. Abschließend werden die Ergebnisse der Projektarbeit vorgestellt und bewertet. 

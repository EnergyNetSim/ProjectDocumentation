\newpage
\section{Ausblick}\label{sec:Ausbl}

Die diskutierten Ergebnisse machen deutlich, dass eine vollständige Betrachtung der Wirtschaftlichkeit energieeffizienter Netzkonzepte weiterer Forschung bedarf. So wurden im Rahmen der vorliegenden Arbeit nur diejenigen Technologien näher betrachtet, die sich gut in das vereinfachte Netzschema des entwickelten Algorithmus einfügen ließen, und es wurden an vielen Stellen Näherungswerte verwendet.
Die durch die Projektgruppe erarbeiteten Ergebnisse können weitere Forschungsvorhaben vereinfachen. So kann der Programmcode	 als Rahmen zur Implementierung des vorgeschlagenen Algorithmus dienen. Ist die Umsetzung in Java-Code erfolgt, können weitere Konzepte zur Energieeinsparung eingefügt werden. Das entwickelte Datenbankschema kann anschließend verfeinert werden, um schrittweise die zur Vereinfachung des Algorithmus getroffenen Annahmen durch reale Werte zu ersetzen. Schließlich müssen, um die Wirtschaftlichkeit und damit die Attraktivität energieeffizienter Netzkonzepte für die Netzbetreiber vollständig erfassen zu können, auch die Investitionskosten in die Berechnung miteinbezogen werden.
Aufgrund der eingangs erwähnten Entwicklungen hinsichtlich des Wachstums des weltweiten Datenvolumens und der steigenden Energiekosten wird die Betrachtung der Effizienz und Wirtschaftlichkeit energiesparender Netzkonzepte auch in Zukunft ein attraktives Forschungsfeld bleiben.
\documentclass[12pt,titlepage]{article}
\usepackage[ngerman]{babel}
\usepackage[utf8]{inputenc}

\usepackage[autostyle=true,german=quotes]{csquotes}%NC: needed for babel

\usepackage[a4paper , lmargin = {2.5cm} , rmargin = {2.5cm} , tmargin =
{2cm} , bmargin = {2cm}]{geometry}

\usepackage[hyphens]{url}
\usepackage[linktocpage=true]{hyperref}

\usepackage{tabularx}

\usepackage{amsmath}
%\numberwithin{equation}{section}

% ### Bibliographical Stuff ###
% Use \textcite{} and \parencite{} with biblatex!
\usepackage[backend=biber,
    style=numeric-comp,%alphabetic-verb,
    sortlocale=de_DE,
    natbib=true,
    url=true, 
    doi=true,
    eprint=false]{biblatex}%NC: replaced bibtex by biber
\addbibresource{literatur.bib}%Dateiname für Quellen einfügen
\usepackage{ragged2e} %für bessere URL formatierung bei printbibliography 

%Bilder einbinden und auf der gleichen seite ein
%Fußnotenzitat einzufügen
%https://en.wikibooks.org/wiki/LaTeX/Importing_Graphics
\usepackage{graphicx}
\DeclareGraphicsExtensions{.pdf,.png,.jpg} %order of preference
\graphicspath{ {./images/} } %NC: Additional path to look for images in (placed higher up to support one location for multiple documents)
%\usepackage{float} %NC: Causes problems. deactivate this and the next
%\usepackage{afterpage} %need to use with: \af­ter­page{\clearpage} https://www.ctan.org/pkg/afterpage?lang=de

\interfootnotelinepenalty=99999 %Seitenumbruch einer Fußzeile verhindern

\usepackage{moreverb} %tabulator formatierung in verbatim umgebung

%Tables:
\newcolumntype{b}{X}
\newcolumntype{s}{>{\hsize=.5\hsize}X}

\newcommand{\bildcite}[6]{
     \begin{figure}[H]
         \includegraphics[width=\textwidth]{Bilder/#1}
         \caption[#2]{#2\footnotemark}
         \label{#3}
     \end{figure}
     \footnotetext{\cite[#6][#5]{#4}}
}

\newcommand{\bild}[3]{
     \begin{figure}[H]
         \includegraphics[width=\textwidth]{Bilder/#1}
         \caption{#2}
         \label{#3}
     \end{figure}
}

\renewcommand{\labelenumi}{\arabic{enumi}.}
\renewcommand{\labelenumii}{\arabic{enumi}.\arabic {enumii}}
\renewcommand{\labelenumiii}{\arabic{enumi}.\arabic{enumii}.\arabic{enumiii}}

\newcommand{\firstpages}{
    % \input{0_Deckblatt.tex}

     \newpage
     \tableofcontents{}
     \addtocontents{toc}{~\hfill\textbf{Seite}\par}

     \newpage
     \listoffigures

     \newpage
     \listoftables
     \newpage
}


\begin{document}
\title{\huge{Wirtschaftlichkeit von energieeffizienten Netzkonzepten} \\ \large{Projektbericht}} 
\author{Veronika Lawrence, Carmen Scheer, Nicholas Cariss, Maximilian Junker,\\ Christian Keck, Stefan Ludowicy, Dominik Schneider} 
\maketitle
\firstpages
%\twocolumn

%START HERE
\section{Einleitung}
Hier muss auch was stehen, oder? 
In der vorliegenden Arbeit...
Sollten wir evtl die Unterkapitel-Aufteilung weg lassen?
\subsection{Motivation}
%TODO Potential alternativer Energiequellen
Steigende Netzlast, steigende Datenvolumen, höherer Energieverbrauch, Industrie 4.0/Internet of Things, Klimawandel, steigende Energiekosten, Wirtschaftlichkeit
%Diagramm Anteil der Telekommunikation am Gesamtenergieverbrauch / Prognose Datenvolumen weltweit
%TODO Dominik


%Text übernommen aus dem Exposé, muss noch ausformuliert werden - VL
%TODO: Bild in Bezug auf Energieverbrauch einfügen, Verbrauch von Kommunikationsnetzen detaillierter darstellen - VL
Nach einer Schätzung des ICT-Analysehauses Gartner erzeugte die Informations- und Kommunikationstechnologie im Jahr 2007 rund 2\% des globalen Ausstoßes an CO2, was dem Ausstoß der Flugzeugbranche entspricht \cite{gartner}. Seit dem ist der Anteil der Bevölkerung weltweit, der das Internet verwendet, von 20,6\% (2007) auf 43,4\% (2015) gestiegen \cite{itu}. Dieses Wachstum wird sich in der nächsten Zukunft nicht verlangsamen. Es ist also unabdingbar, den CO2-Ausstoß, der durch ICT im Allgemeinen verursacht wird, drastisch zu verringern.

\begin{figure}[!ht]
	\centering
	\includegraphics[width=0.8\textwidth]{statista}
	\caption{Verteilung der weltweiten Energieerzeugung nach Energieträger im Jahr 2013 (Quelle: statista)}
	\label{fig:statista}
\end{figure}

Da aller Bemühungen zum Trotz die fossilen Energieträger Erdöl, Kohle und Erdgas weiterhin ca. 81\% der weltweiten Energieerzeugung ausmachen \cite{statista}, ist die Reduktion des Energieverbrauchs von ICT-Infrastruktur ein wichtiger Ansatz, um den CO2-Ausstoß zu verringern. Natürlich sind neben dem Bemühen, ICT grüner zu gestalten, wirtschaftliche Bedenken ein weiterer wichtiger Treiber für das Streben nach mehr Energieeffizienz. Langfristig steigende Energiekosten (\cite{iea2015}, S. 40f) sind ein großes Risiko für ICT-Provider weltweit, die insgesamt mit steigenden Kosten zu kämpfen haben.

Die vorliegende Arbeit beschränkt sich auf die ökonomischen Zwänge, die sich aus ineffizienten Netzen ergeben. Es soll eine Software entwickelt werden, die es ermöglicht, zwei hypothetische Netze miteinander zu vergleichen, zum einen ein konventionelles Netz, wie es heutzutage Stand der Technik ist, zum anderen ein energieeffizientes Netz, das die vorhandenen Technologien und Konzepte zur Effizienzsteigerung sinnvoll einsetzt. Anhand des berechneten Energieverbrauchs der beiden Netze wird das Energiesparpotenzial sowie die möglichen Kosteneinsparungen durch den Betrieb des energieeffizienten Netzes ausgegeben.


\subsection{Ziele}
Wie viel Energie/Kosten kann durch Verwendung energieeffizienter Technologien eingespart werden? 

Zu diesem Zweck soll eine Software erstellt werden, die unterstützt.


\section{Stand der Forschung}
Definition Wirtschaftlichkeit, Carriernetz

Betriebskosten, Wirtschaftlichkeit, etc.
CAPEX OPEX
Wirtschaftlichkeit
Definieren
Wir betrachten einen Teil der Betriebskosten, nämlich: .....

\subsection{Carrier-Netzwerke}

%Text übernommen aus dem Exposé, muss noch ausformuliert werden - VL
%
%TODO: Mehr beschreiben, was schon getan wurde
%TODO: Verschiedene Methoden gegenüberstellen

Gegenstand der Projektarbeit bildet ein hinsichtlich der Energieeffizienz und Wirtschaftlichkeit zu optimierendes Carrier-Netzwerk.

Unter einem Carrier "`versteht man […] eine Gesellschaft, die mindestens drei Übertra\-gungs\-wege betreibt, die über eine Vermittlungsstelle miteinander verbunden sein müssen"' (aus: \cite{carrier}). Ein Carrier-Netzwerk stellt somit physikalische Transportwege und -verfahren zur Verfügung und bildet die Grundlage für sogenannte Value Added Services von Providern, welche auf den Carrier-Diensten aufsetzen \cite{fassnacht}. "`Bei den TK-Transportwegen unterscheidet man leitergebundene Verbindungen auf der Basis von Kupferkabeln oder Lichtwellenleitern sowie Funkverbindungen wie Satellitenverbindungen, Richtfunkstrecken und Rundfunkverbindungen"' (Aus: Ebd.).%TODO: Zitat korrigieren

Somit umfasst ein Carrier-Netzwerk nicht nur das physikalische Backbone-Netz, sondern auch das Zugangs- und Aggregationsnetzwerk. Im Rahmen des Projektes entfällt die Betrachtung der Optimierungspotentiale des Zugangsnetzes zugunsten einer aus\-führ\-lich\-eren Simulation von wirtschaftlichen und energieeffizienten Netzkonzepten im Backbone- und Aggregationsnetzwerk. So bildet der Broadband Network Gateway (BNG) die niedrigste Netzelementebene. Multi-Service Access Nodes (MSAN), Digital Subscriber Line Access Multiplexer (DSLAM), enhanced NodeBs (eNodeB) und weitere Netzelemente der nächsttieferen Hierarchiestufe fallen somit aus der Betrachtung.
\subsection{Möglichkeiten der Energieeinsparung in Netzwerken}
%Evtl mit dem nächsten Subchapter zusammenlegen? Hier nicht schreiben über unsere Simulation. Es geht nur darum, was andere schon erforscht haben.
Um den Energieverbrauch im Carrier-Netzwerk berechnen zu können, bedarf es an Werten der im Backbone verwendeten Netzkomponenten. Innerhalb der Simulation wird für die Berechnung des Gesamtverbrauchs auf die definierten Werte zurückgegriffen, die in einer Datenbank gespeichert sind. Ziel dieser Arbeit ist die Simulation und Berechnung des gesamtheitlichen Energieverbrauchs. Deshalb verwendet die Berechnung die von den beiden Wissenschaftler Ward Van Heddeghem und Filip Idzikowski in ihrer Veröffentlichung \cite{vanhedde} zusammengetragenen Werte. Die Quelle beinhaltet zum einen das analytische Model der Berechnung und zum anderen ein Datenblatt \cite{vanhsheet} des Energieverbrauchs der unterschiedlichen Hersteller. Das Datenblatt gliedert die Geräte nach den Unterschiedlichen Layer IP/MPLS, Ethernet, OTN, WDM - (OSI-Layer: 3-2-1-1). Zu beachten ist beim Verwenden der Werte, dass es sich um Werte unter typischen Lastbedingungen handelt, die sich nach der Kapazität der Komponente richtet und nicht nach dem aktuellen Durchsatz. Des Weiteren geben die Werte nur dein Stromverbrauch für den Betrieb an, ein Verbrauch für Kühlung o.Ä. ist nicht enthalten.
Der Gesamtverbrauch des Core-Networks ergibt sich aus der Summe alle Verbrauche der einzelnen Schichten. 

\begin{equation}
P_{core} = P_{ip} + P_{ethernet} + P_{otn} + P_{wdm}
\end{equation}


\subsection{Energie-effiziente Technologien}
Der aktuelle Stand der Forschung bietet verschiedene Konzepte und Technologien für das Einsparen von Energie in Carrier-Netzen. In dieser Arbeit sollen ausgewählte Konzepte und Technologien genutzt werden, um die Wirtschaftlichkeit von energieeffizienten Netzen zu analysieren. Betrachtet wurden im Rahmen der Literaturrecherche die folgenden Konzepte und Technologien: 

Das erste Konzept sieht eine Vereinfachung des Netzes vor, welche durch eine geographische Aufteilung des Netzes in Submodale (Global - Kontinental - National - Regional - Zugang) erfolgen kann \cite{aleksic2014}.

Das zweite Konzept befasst sich mit der dynamischen Abschaltung unterlasteter Netzkomponenten. Tageszeitabhängige Schwankungen des Traffics in Netzen ermöglichen eine individuelle und dynamische Abschaltung von Switches und Links unter der Be\-rück\-sich\-ti\-gung der QoS-Bedingungen \cite{aleksic2013}. Dabei werden Algorithmen zur Identifikation von individuell abschaltbaren Links verwendet \cite{fisher}.

Bei den Technologien beschränkt sich diese Arbeit auf optische leitungsvermittelnde Switche und das Hybrid Optical Switching (HOS). Optische leitungsvermittelnde Switche basieren auf mikro-elektromechanischen Systemen, die die geringste Menge an Energie benötigen und eine hohe Portanzahl besitzen. Das Hybrid Optical Switching verwendet eine Kombination aus optischen und elektronischen Netzknoten, die optische Leitungen, Bursts, und Pakete effizient Switches können. Durch die Kombination von langsamen und schnellen Switches, können Wellenlängenbereiche dynamisch zwischen zwei Switches geändert werden. Das temporäre Abschalten von ungenutzten Ports des schnellen Switches ermöglicht eine Energieeinsparung.  \cite{aleksic2013}

\section{Problemstellung}
% Wir haben gezeigt, was schon getan wurde. Hier jetzt beschreiben, was noch zu tun ist (und warum?)

\section{Methoden/Vorgehen}




\subsection{Modellierung}
Zwei Netze modelliert (Quellen), um sie gegenüberstellen zu können. Techniken aus Kapitel "Energieeffiziente Technologien" eingesetzt.
Aufgrund der Recherche ist Simulation nötig.

\subsection{Simulationsansatz}
%Brauchen wir hier subsubsections oder machen wir das als subsections (gleichwertig zu modellierung)?
\subsubsection{Softwareentwicklung}

Anforderungen an eine mögliche Simulationssoftware erarbeitet
Entscheidung/Festlegung Entwicklungmodell / Technologiestack / Systemlandschaft, weil....
Start Softwareentwicklung
	- Requirements
	- Spezifikation
	- Design
Implementierung
\subsubsection{Entwicklung des Routing-Algorithmus}

\subsubsection{Komplexität bei der Entwicklung}

\section{Ergebnisse, Beitrag, Diskussion}
\subsection{Algorithmus}
\subsection{Software}
\subsection{Abschätzung des Energieverbrauchs}
%Für die zwei Beispiel-Szenarien


\section{Präsentation und Auswertung der Ergebnisse}

\section{Diskussion}

\section{Ausblick}

\newpage
\printbibliography 

\end{document}